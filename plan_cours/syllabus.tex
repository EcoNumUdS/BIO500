\documentclass[12]{article}
\usepackage[utf8]{inputenc}
\usepackage{authblk} 
\usepackage{geometry}
\usepackage{amsmath} 
\usepackage{hyperref}
\usepackage[french]{babel}
\usepackage{url}

\geometry{letterpaper,margin=2.5cm}

\hypersetup
{
    colorlinks = true, linkcolor = blue, citecolor = blue, urlcolor = blue,
}

%\def\labelitemi{$\bullet$}

\title{BIO-500 \\ Méthodes en écologie computationelle}
\date {\today}
\author {Dominique Gravel}

\affil {Département de biologie \\
Université de Sherbrooke \\ 
Local D8-3066 \\ 
819-821-8000 \#66589}
\affil {\url{dominique.gravel@usherbrooke.ca}}

\begin{document}

	\maketitle

	%-----------------------------
	\section*{Objectif général}

	Les outils informatiques sont utilisés de façon croissante en écologie,
	que ce soit pour la réalisation d'analyses spatiales, statistiques ou pour
	la gestion de bases de données. On exige de plus en plus la transparence et
	la reproductibilité des études scientifiques et d'évaluations environnementales. 
	Au terme de ce cours, l'étudiant sera en mesure de réaliser l'ensemble de la 
	séquence d'une étude d'écologie en respectant les standards de gestion,  d'analyse et de présentation des données. Le cours portera sur la réalisation d'un projet intégrateur, de la collecte des données à la production du rapport final. 

	%-----------------------------
	\section*{Objectifs spécifiques}

	Au terme de ce cours, l'étudiant sera en mesure de: 

	\begin{itemize}
	\renewcommand{\labelitemi}{$\bullet$}

	\item Planifier une base de données et la préparation de formulaires pour leur acquisition ; 

	\item Programmer et interroger une base de données relationnelle ;

	\item Compiler et exécuter un script au moyen de makefile ;

	\item Représenter visuellement les données au moyen de R ;

	\item Préparer un rapport d'étude au moyen de LaTeX ;

	\item Utiliser un système de contrôle de version pour le suivi des modifications sur du code ;

	\item Porter un regard critique sur la reproductibilité et la transparence d'études scientifiques ;

	\end{itemize}

	%-----------------------------
	\section*{Pré-requis}

	Ce cours obligatoire est offert aux étudiants en fin de programme de
	baccalauréat en biologie, concentration écologie. Aucun pré-requis n'est
	exigé pour ce cours. 

    %-----------------------------     

    \section*{Approche pédagogique} 

Les séances seront constituées de courtes leçons magistrales sur des notions
de bases sur les différents outils utilisés, entre-coupées d'exercices
spécifiques destinés à pratiquer les éléments enseignés. Les séances seront
complémentés de discussions sur les enjeux de la reproductibilité en science.
Les séances se conclueront sur la réalisation  d'un exercice intégrateur à
compléter à la maison. L'apprentissage portera sur la réalisation d'un projet
de session où les étudiants seront responsables de l'ensemble des étapes d'une
étude en écologie. Le travail sera réalisé par blocs, au fur et à mesure de la
présentation du matériel.


	L'ensemble du matériel du cours sera disponible sur un dépôt git à l'adresse :\\
	https://github.com/EcoNumUdS/BIO500.git

	%-----------------------------
	\section*{Contenu}

	\subsection*{Bloc 1 (3 séances): Planification de la collecte et organisation des données}

	\begin{itemize}
	\renewcommand{\labelitemi}{$\bullet$}	
		\item Types de données ;
		\item Formulaires de saisie
		\item Bases de données relationnelles SQL ;
		\item Requêtes ;
	\end{itemize}


	\subsection*{Bloc 2 (1 séance): Outils pour une science reproductible et transparente}

	\begin{itemize}
	\renewcommand{\labelitemi}{$\bullet$}	
		\item UNIX et le makefile ;
		\item Système de contrôle de version git ;
	\end{itemize}


	\subsection*{Bloc 3 (2 séances): Visualisation des données au moyen de R}

	\begin{itemize}
	\renewcommand{\labelitemi}{$\bullet$}	
		\item Fonctions graphiques de base et paramètres graphiques ;
		\item Packages R spécialisés ;
	\end{itemize}


	\subsection*{Bloc 4 (3 séances): Communication scientifique au moyen de LaTeX}

	\begin{itemize}
	\renewcommand{\labelitemi}{$\bullet$}	
		\item Rédaction de rapports et articles scientifiques ;
		\item Gestion des références ;
		\item Utilisation de Beamer pour les présentations ;
	\end{itemize}


	%-----------------------------
	\section*{Évaluation}

	L'évaluation porte sur la réalisation d'un travail de session (75\%), réalisé en équipe de 4 personnes. Le travail sera divisé en 3 étapes réparties au cours de la session. L'évaluation finale (25\%) portera sur la rédaction d'un essai de 1500 mots sur les enjeux de reproductibilité en science expérimentale.  


\end{document}
