\documentclass[12]{article}
\usepackage[utf8]{inputenc}
\usepackage{array}
\usepackage{caption}
\usepackage{authblk}
\usepackage{geometry}
\usepackage{amsmath}
\usepackage{hyperref}
\usepackage{multirow} % Pour les tableaux
\usepackage{longtable} % Pour les tableaux
\usepackage[french]{babel}
\usepackage{url}
\usepackage[french]{isodate}

\geometry{letterpaper,margin=2.5cm}

\hypersetup
{
    colorlinks = true, linkcolor = blue, citecolor = blue, urlcolor = blue,
}

%\def\labelitemi{$\bullet$}

\title{BIO500 : Méthodes en écologie computationelle}
\date {Hiver 2025}
\author {Victor Cameron}

\affil {Département de biologie \\
Université de Sherbrooke \\ 
Local D8-0012 \\ 
819-821-8000 \#61928}
\affil {\url{victor.cameron@usherbrooke.ca}}


\begin{document}

	\maketitle

	%-----------------------------
	\section*{Renseignements généraux\footnote{Ce plan de cours est soumis au \textit{Règlement facultaire d'évaluation des apprentissages des étudiantes et des étudiants} de la faculté des sciences de l'Université de Sherbrooke et y est conforme.}} 
        
        \begin{center}
			\begin{tabular}{ m{22em} m{24em} }
				\hline
				\hline
				\textbf{Nombre de crédits} & 2 \\ 
				\hline
				\textbf{Cours préalables} & Aucun \\
				\hline
				\textbf{Lieu du cours} & D7-2021 \\
				\hline
				\textbf{Jours et heures des cours} & 11 et 18 février, 11, 18 et 25 mars, 1 et 8 avril \\ & Tous les cours sont le mardi de 8h30 à 11h30 \\
				\hline
				\textbf{Session} & Hiver 2025 \\
				\hline
				\textbf{Date de début} & 11 février \\
				\hline
				\textbf{Date de fin} & 8 avril \\
				\hline
				\textbf{Date de remise de l'évaluation finale} & 15 et 22 avril à 16h00 \\
				\hline
				\textbf{Date limite de retrait} & 17 février \\
				\hline
				\textbf{Date limite d'abandon} & 24 mars \\
				\hline
				\textbf{Assistant à l'enseignement} & Zacharie Sclazo \\ & Local D8-0022 \\ & \url{Zacharie.Scalzo@usherbrooke.ca} \\
				\hline
				\hline
			\end{tabular}
        \end{center}

	%-----------------------------
	\section*{Objectif général}

	Les outils informatiques sont utilisés de façon croissante en écologie,
	que ce soit pour la réalisation d'analyses spatiales, statistiques ou pour
	la gestion de bases de données. On exige de plus en plus la transparence et
	la reproductibilité des études scientifiques et d'évaluations environnementales. 

	Au terme de ce cours, l'étudiant sera en mesure de réaliser l'ensemble de la 
	séquence d'une étude d'écologie en respectant les standards de gestion,  d'analyse et de présentation des données. Le cours portera sur la réalisation d'un projet intégrateur, de la collecte des données à la production du rapport final. 

	\newpage
	
	%-----------------------------
	\section*{Objectifs spécifiques}

	Au terme de ce cours, l'étudiant sera en mesure de: 

	\begin{itemize}
	\renewcommand{\labelitemi}{$\bullet$}

	\item Planifier une base de données et la préparation de formulaires pour leur acquisition ; 

	\item Programmer et interroger une base de données relationnelle ;

	\item Compiler et exécuter un projet au moyen de la librairie trargets;

	\item Représenter visuellement les données au moyen de R ;

	\item Préparer un rapport d'étude au moyen de RMarkdown ;

	\item Utiliser un système de contrôle de version pour le suivi des modifications sur du code ;

	\item Porter un regard critique sur la reproductibilité et la transparence d'études scientifiques ;

	\end{itemize}

	%-----------------------------
	\section*{Pré-requis}

	Un ordinateur portable personnel est requis pour ce cours.\\

	Ce cours obligatoire est offert aux étudiants en fin de programme de
	baccalauréat en biologie, concentration écologie. Le cours 
	\textit{BIO109 : Introduction à la programmation scientifique} est préalable à celui-ci.

    %-----------------------------     

    \section*{Approche pédagogique} 

	Les séances seront constituées de courtes leçons magistrales sur des notions
	de bases sur les différents outils utilisés, entre-coupées d'exercices
	spécifiques destinés à pratiquer les éléments enseignés. Les séances seront
	complémentés de discussions sur les enjeux de la reproductibilité en science.
	Les séances se conclueront sur la réalisation  d'un exercice intégrateur à
	compléter à la maison. L'apprentissage portera sur la réalisation d'un projet
	de session où les étudiants seront responsables de l'ensemble des étapes d'une
	étude en écologie. Le travail sera réalisé par blocs, au fur et à mesure de la
	présentation du matériel.


	L'ensemble du matériel du cours sera disponible sur un dépôt git à l'adresse :\\
	\url{https://github.com/EcoNumUdS/BIO500.git}

	%-----------------------------
	\section*{Contenu}

% Please add the following required packages to your document preamble:
% \usepackage{multirow}
% \usepackage{longtable}
% Note: It may be necessary to compile the document several times to get a multi-page table to line up properly
\begin{longtable}[c]{| p{0.2\linewidth} | p{0.25\linewidth} | p{0.3\linewidth} | p{0.25\linewidth} | }
\hline
Dates &
  Contenus &
  Activités &
  Évaluations \\ \hline
\endhead
%
\multirow{3}{*}{\begin{tabular}[c]{@{}l@{}}\textbf{Bloc 1}\\ 11 février au 17 mars\\ \\ Planification et\\organisation des\\ données\end{tabular}} &
  \textbf{Introduction} &
  \begin{tabular}[c]{@{}l@{}}\textbf{Séance 1} (11 février)\\ \\ \textit{Travaux}\\ - Explorer et choisir\\ un jeu de données\\- Préparer les arguments pour le\\débat\\ \\ \textit{Lectures}\\ - Baker. 2016. Is there a\\reproducibility crisis ?\\ - Poisot et al. 2014. Moving\\toward a sustainable ecological\\science: don't let ecological data\\go to waste!\\ - Mills et al. 2015. Archiving\\Primary Data: Solutions for\\Long-term Studies.\end{tabular} &
   \\ \cline{2-4} 
 &
  \begin{tabular}[c]{@{}l@{}}\textbf{La gestion des données}\\ \textbf{biologiques}\\ - Types de données\\ - Formulaires de saisie\\ - Base de données\\relationnelles SQL\end{tabular} &
  \begin{tabular}[c]{@{}l@{}}\textbf{Séance 2} (18 février)\\ \textbf{Débat sur le partage de}\\ \textbf{données}\\ \\ \textit{Travaux}\\ - Plan d'assemblage des\\ données\\ - Nettoyer et valider\\ les données\\ - Concevoir et scripter une base\\de données\end{tabular} &
  \begin{tabular}[c]{@{}l@{}}\textbf{Évaluation formative}\\ Évaluation par les pairs\\des scripts d'assemblage,\\ de nettoyage et de\\ validation des données\\à remettre le 10 mars\\ sur Moodle\end{tabular}
  \\ \cline{2-4} 
 &
  \begin{tabular}[c]{@{}l@{}}\textbf{La gestion des données}\\ \textbf{biologiques}\\ - Bases de données\\relationnelles SQL\\ - Requêtes\end{tabular} &
  \begin{tabular}[c]{@{}l@{}}\textbf{Séance 3} (11 mars)\\ Reproductibilité\\ \\ \textit{Travaux}\\ - Scripter le traitement, la\\création de la base de données et\\l'injection de données\\ - Préparer les requêtes et les\\résultats\\ \\ \textit{Lectures}\\ - Biswas.2023. ChatGPT for\\Research and Publication:\\A Step-by-Step Guide.\\ - Nature Machine Learning.\\2023. The AI writing on the wall.\\ - Wilkinson et al. 2016.\\The FAIR Guiding Principles for\\scientific data management\\and stewardship.\end{tabular} &
   \\ \hline
\multirow{2}{*}{\begin{tabular}[c]{@{}l@{}}\textbf{Bloc 2}\\ 18 au 31 mars\\ \\ Outils pour une science\\reproductible\end{tabular}} &
  \begin{tabular}[c]{@{}l@{}}\textbf{Les outils pour la}\\ \textbf{reproductibilité}\\ - Système de contrôle de\\version Git\\ - Gestion des conflits\end{tabular} &
  \begin{tabular}[c]{@{}l@{}}\textbf{Séance 4} (18 mars)\\ Débat sur l'impact d'algorithmes\\comme ChatGPT sur la science\\ et sa transparence\\ \\ \textit{Lectures}\\ - Milcu et al. 2018. Genotypic\\variability enhances the\\ reproducibility of an ecological\\ study\end{tabular} &
   \\ \cline{2-4} 
 &
  \begin{tabular}[c]{@{}l@{}}\textbf{Les outils pour la}\\ \textbf{reproductibilité}\\ - Cahier de laboratoire\\ avec RMarkdown\\ - Librairie targets\end{tabular} &
  \begin{tabular}[c]{@{}l@{}}\textbf{Séance 5} (25 mars)\\ \textbf{Discussion sur les liens entre}\\ \textbf{la variabilité et la}\\ \textbf{reproductibilité}\\ \\ \textit{Travaux}\\ - Créer un cahier de laboratoire\\et le sauver sur GitHub\\ - Construire un script target pour\\suivre l'évolution du travail \end{tabular} &
  \begin{tabular}[c]{@{}l@{}}\textbf{Évaluation formative}\\ Évaluation par les pairs\\des scripts de création\\ de la base de données\\à remettre le 31 mars\\ sur Moodle\\ \\ \textbf{Essai (25\%)}\\ Rédaction d'un essai sur les\\enjeux de reproductibilité en\\science à remettre le 15 avril\\à 16h sur Moodle\end{tabular} 
  \\ \hline
\begin{tabular}[c]{@{}l@{}}\textbf{Bloc 3}\\ 1 au 7 avril\\ \\ Visualisation des\\données\end{tabular} &
  \begin{tabular}[c]{@{}l@{}}\textbf{Visualisation des}\\ \textbf{données au moyen de R}\\ - Fonctions graphiques de\\base et paramètres\\graphiques\\ - Librairies R spécialisées\end{tabular} &
  \begin{tabular}[c]{@{}l@{}}\textbf{Séance 6} (1er avril)\\ \\ \textit{Travaux}\\ - Identifier les questions de\\recherche\\ - Compléter l'analyse au moyen\\de visualisations\\ \\ \textit{Lectures}\\ - Pennisi. 2020. Spider biologist\\denies suspicions of widespread\\data fraud in his animal\\ personality research\\ - Lawkowski. 2020. What to do\\when you don't trust your\\data anymore\end{tabular} &
   \\ \hline
\begin{tabular}[c]{@{}l@{}}\textbf{Bloc 4}\\ 8 au 22 avril\\ \\ Communication\\scientifique au moyen\\de RMarkdown\end{tabular} &
  \begin{tabular}[c]{@{}l@{}}\textbf{Documents dynamiques}\\ \textbf{avec RMarkdown}\\ - Rédaction de rapports\\scientifiques\\ - Gestion des références\end{tabular} &
  \begin{tabular}[c]{@{}l@{}}\textbf{Séance 7} (8 avril)\\ \textbf{Discussion sur la fraude en}\\ \textbf{science et sa prévention}\end{tabular} &
  \begin{tabular}[c]{@{}l@{}}\textbf{Évaluation terminale}\\ \textbf{(75\%)}\\ Travail de session : Écrire\\un rapport de votre analyse\\des données sous forme\\ d'article scientifique\\ \\Tous les scripts sont à\\remettre sous la forme\\d'un dépôt GitHub\end{tabular} \\ \hline
\end{longtable}

\newpage
	%-----------------------------
	\section*{Évaluation}

	L'évaluation porte sur la réalisation d'un travail de session (75\%), réalisé en équipe de 4 personnes. Le travail sera divisé en 3 étapes réparties au cours de la session. L'évaluation finale portera aussi sur la rédaction d'un essai (25\%) sur les enjeux de reproductibilité en science. L'essai doit être déposé au plus tard le 15 avril à 16:00 et le travail de session le 22 avril 2024 à 16:00. La pénalité est de 10\% par journée de retard.  

	\subsection*{Modalités de remise}

	Les travaux devront tous être remis sur Moodle. Aucun travail ne sera
	accepté par courrier électronique.

	\subsection*{Modalités de correction et de notation pour l'évaluation terminale}

	La note obtenue pour l’ensemble des travaux sera convertie en fonction des
	cotes proposées par la Politique d’évaluation de l’Université de Sherbrooke.
	La notation définitive sera exprimée en conformité avec le règlement de la
	Faculté des sciences de l’Université de Sherbrooke, soit à partir du tableau
	suivant :
	
	\begin{center}
		\begin{table}[h]
        \begin{tabular}{| p{0.2\linewidth} | p{0.2\linewidth} | p{0.2\linewidth} | p{0.2\linewidth} | p{0.2\linewidth} | } 
        \hline
		% Header ------------
        \textbf{Excellent} \linebreak A+, A, A- &
		\textbf{Très bien}\hfill\hfill \linebreak B+, B, B- & 
		\textbf{Bien} \linebreak C+, C, C- & 
		\textbf{Passable} \linebreak D+, D &
		\textbf{Échec} \linebreak E \\ [0.5ex] 
        \hline\hline
		% Body ------------
		% A+ : 91\% et plus \hfill\hfill \linebreak A : 88 à 90\% \hfill\hfill \linebreak A- : 85 à 87\% \hfill\hfill & 
		% B+ : 82 à 84\% \hfill\hfill \linebreak B : 79 à 81\% \hfill\hfill \linebreak B- : 76 à 78\% \hfill\hfill & 
		% C+ : 73 à 75\% \hfill\hfill \linebreak C : 70 à 72\% \hfill\hfill \linebreak C- : 67 à 69\% \hfill\hfill & 
		% D+ : 64 à 66\% \hfill\hfill \linebreak D : 60 à 63\% \hfill\hfill & 
		% E : 0 à 59\% \hfill \\ 
		% \hline
		% \hline
		% Footer ------------
        \multicolumn{2}{l}{W : échec par abandon\hfill\hfill \linebreak
            AB* : abandon\hfill\hfill \linebreak} &
        \multicolumn{2}{l}{
            IN** : Incomplet\hfill\hfill \linebreak 
            R : réussite\hfill\hfill} \\
		% \hline
		\hline
		\end{tabular}
		\caption*{* La mention AB est consignée seulement si l’étudiante ou l’étudiant abandonne le cours avant la date
		limite d'abandon. Si la date limite d’abandon est dépassée, la mention au relevé de notes de l’étudiante
		ou de l’étudiant sera W.\\
		** La metion IN est utilisée au relevé de notes pour les activités
		pédagogiques lorsque, pour des motifs acceptés par la faculté ou le
		centre universitaire de formation, l’étudiante ou l’étudiant n’a pas
		satisfait à toutes les exigences. Est remplacée par la note W (échec
		par abandon) au relevé de notes du trimestre au cours duquel prend fin
		le délai accordé si l’activité n’a pas été complétée.}
		\end{table}
	\end{center}

    La note finale du cours sera remise au plus tard deux semaines après le
	dépôt de l'évaluation finale.

    \subsection*{Appréciation de la qualité de la langue}

    En conformité avec l'article 17 du règlement facultaire d'évaluation
    des apprentissages des étudiantes et des étudiants, la qualité du français
    écrit dans l’évaluation peut être pis en compte. Tout travail non conforme
    aux exigences quant à la qualité du français écrit et aux normes de
    présentation  peut retourné à l'étudiante ou à l'étudiant et peut aussi
    entrainer la perte de points pour une mauvaise qualité du français écrit.
    La qualité du français peut compter jusqu'à 5\% des points alloués à
    l'évaluation.

	\subsection*{Antiplagiat}

	Le plagiat correspond à un délit relatif aux études et <<[...] désigne tout acte 
	trompeur ou toute tentative de commentre un tel acte, quant au rendement
	scolaire ou une exigence relative à une activité pédagogique [...]>> 
	(Règlement 2575-009). L'intégrité intellectuelle est valurisée et sera monitorée, 
	un manquement à ce règlement entrainera des sanctions disciplinaires.

	\subsection*{Ressources d'aide}

	Santé mentale 
	\begin{itemize}
		\item Service de psychologie et d'orientation : 819 821-7666, spo@USherbrooke.ca
		\item Programme mieux-être-Ressources en santé mentale et PAE Optima : Urgence 24/7 1 833 851-1363
		\item La Pair-Mission : pair-mission@USherbrooke.ca
	\end{itemize}
		
	Respects et droits de la personne
	\begin{itemize}
		\item Équipe-conseil en matière de respect des personnes : 819 821-7410, Respect@USherbrooke.ca
		\item Ombudsman des étudiantes et étudiants : 819 821-7706, ombudsman@USherbrooke.ca
	\end{itemize}

\end{document}
