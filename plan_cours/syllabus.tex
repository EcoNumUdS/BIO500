\documentclass[12]{article}
\usepackage[utf8]{inputenc}
\usepackage{array}
\usepackage{caption}
\usepackage{authblk}
\usepackage{geometry}
\usepackage{amsmath}
\usepackage{hyperref}
\usepackage[french]{babel}
\usepackage{url}
\usepackage[french]{isodate}

\geometry{letterpaper,margin=2.5cm}

\hypersetup
{
    colorlinks = true, linkcolor = blue, citecolor = blue, urlcolor = blue,
}

%\def\labelitemi{$\bullet$}

\title{BIO-500 : Méthodes en écologie computationelle}
\date {Hiver 2023}
\author {Victor cameron}

\affil {Département de biologie \\
Université de Sherbrooke \\ 
Local D8-0012 \\ 
819-821-8000 \#61928}
\affil {\url{victor.cameron@usherbrooke.ca}}


\begin{document}

	\maketitle

	%-----------------------------
	\section*{Renseignements généraux\footnote{Ce plan de cours est soumis au \textit{Règlement facultaire d'évaluation des apprentissages des étudiantes et des étudiants} de la faculté des sciences de l'Université de Sherbrooke et y est conforme.}} 
        
        \begin{center}
			\begin{tabular}{ m{22em} m{24em} }
				\hline
				\hline
				\textbf{Nombre de crédits} & 2 \\ 
				\hline
				\textbf{Cours préalables} & Aucun \\
				\hline
				\textbf{Lieu du cours} & D7-2023 \\
				\hline
				\textbf{Jours et heures des cours} & 14 février, 7, 14, 21, 28 mars, 4 et 11 avril \\ & Tous les cours sont le mardi de 8h30 à 11h30 \\
				\hline
				\textbf{Session} & Hiver 2023 \\
				\hline
				\textbf{Date de début} & 14 février \\
				\hline
				\textbf{Date de fin} & 11 avril \\
				\hline
				\textbf{Date de remise de l'évaluation finale} & 22 avril à 16h00 \\
				\hline
				\textbf{Date limite de retrait} & 6 mars \\
				\hline
				\textbf{Date limite d'abandon} & 28 mars \\
				\hline
				\textbf{Assistant à l'enseignement} & Benjamin Mercier \\ & Local D8-0022 \\ & \url{benjamin.b.mercier@usherbrooke.ca} \\
				\hline
				\hline
			\end{tabular}
        \end{center}

	%-----------------------------
	\section*{Objectif général}

	Les outils informatiques sont utilisés de façon croissante en écologie,
	que ce soit pour la réalisation d'analyses spatiales, statistiques ou pour
	la gestion de bases de données. On exige de plus en plus la transparence et
	la reproductibilité des études scientifiques et d'évaluations environnementales. 

	Au terme de ce cours, l'étudiant sera en mesure de réaliser l'ensemble de la 
	séquence d'une étude d'écologie en respectant les standards de gestion,  d'analyse et de présentation des données. Le cours portera sur la réalisation d'un projet intégrateur, de la collecte des données à la production du rapport final. 

	%-----------------------------
	\section*{Objectifs spécifiques}

	Au terme de ce cours, l'étudiant sera en mesure de: 

	\begin{itemize}
	\renewcommand{\labelitemi}{$\bullet$}

	\item Planifier une base de données et la préparation de formulaires pour leur acquisition ; 

	\item Programmer et interroger une base de données relationnelle ;

	\item Compiler et exécuter un projet au moyen de la librairie trargets;

	\item Représenter visuellement les données au moyen de R ;

	\item Préparer un rapport d'étude au moyen de RMarkdown ;

	\item Utiliser un système de contrôle de version pour le suivi des modifications sur du code ;

	\item Porter un regard critique sur la reproductibilité et la transparence d'études scientifiques ;

	\end{itemize}

	%-----------------------------
	\section*{Pré-requis}

	Un ordinateur portable personnel est requis pour ce cours.\\

	Ce cours obligatoire est offert aux étudiants en fin de programme de
	baccalauréat en biologie, concentration écologie. Le cours 
	\textit{BIO109 : Introduction à la programmation scientifique} est préalable à celui-ci.

    %-----------------------------     

    \section*{Approche pédagogique} 

	Les séances seront constituées de courtes leçons magistrales sur des notions
	de bases sur les différents outils utilisés, entre-coupées d'exercices
	spécifiques destinés à pratiquer les éléments enseignés. Les séances seront
	complémentés de discussions sur les enjeux de la reproductibilité en science.
	Les séances se conclueront sur la réalisation  d'un exercice intégrateur à
	compléter à la maison. L'apprentissage portera sur la réalisation d'un projet
	de session où les étudiants seront responsables de l'ensemble des étapes d'une
	étude en écologie. Le travail sera réalisé par blocs, au fur et à mesure de la
	présentation du matériel.


	L'ensemble du matériel du cours sera disponible sur un dépôt git à l'adresse :\\
	\url{https://github.com/EcoNumUdS/BIO500.git}

	%-----------------------------
	\section*{Contenu}

	\subsection*{Bloc 1 (3 séances): Planification de la collecte et organisation des données}

	\begin{itemize}
	\renewcommand{\labelitemi}{$\bullet$}	
		\item Types de données ;
		\item Formulaires de saisie ;
		\item Bases de données relationnelles SQL ;
		\item Requêtes ;
	\end{itemize}


	\subsection*{Bloc 2 (1 séance): Outils pour une science reproductible et transparente}

	\begin{itemize}
	\renewcommand{\labelitemi}{$\bullet$}	
		\item Système de contrôle de version git ;
		\item Librairie targets ; 
		\item Cahier de laboratoire avec RMarkdown
	\end{itemize}


	\subsection*{Bloc 3 (2 séances): Visualisation des données au moyen de R}

	\begin{itemize}
	\renewcommand{\labelitemi}{$\bullet$}	
		\item Fonctions graphiques de base et paramètres graphiques ;
		\item Libraries R spécialisés ;
	\end{itemize}


	\subsection*{Bloc 4 (1 séance): Communication scientifique au moyen de Rmarkdown}

	\begin{itemize}
	\renewcommand{\labelitemi}{$\bullet$}	
		\item Rédaction de rapports et articles scientifiques ;
		\item Gestion des références ;
	\end{itemize}


	%-----------------------------
	\section*{Évaluation}

	L'évaluation porte sur la réalisation d'un travail de session (75\%), réalisé en équipe de 4 personnes. Le travail sera divisé en 3 étapes réparties au cours de la session. L'évaluation finale (25\%) portera sur la rédaction d'un essai de 1200 mots sur les enjeux de reproductibilité en science expérimentale. Le travail de session et l'essai doivent être déposés au plus tard le vendredi 22 avril 2020 à 16:00. La pénalité est de 10\% par journée de retard.  

	\subsection*{Modalités de remise}

	Les travaux devront tous être remis sur Moodle. Aucun travail ne sera
	accepté par courrier électronique.

	\subsection*{Modalités de correction et de notation pour l'évaluation terminale}

	La note obtenue pour l’ensemble des travaux sera convertie en fonction des
	cotes proposées par la Politique d’évaluation de l’Université de Sherbrooke.
	La notation définitive sera exprimée en conformité avec le règlement de la
	Faculté des sciences de l’Université de Sherbrooke, soit à partir du tableau
	suivant :
	
	\begin{center}
		\begin{table}[h]
        \begin{tabular}{| p{0.2\linewidth} | p{0.2\linewidth} | p{0.2\linewidth} | p{0.2\linewidth} | p{0.2\linewidth} | } 
        \hline
		% Header ------------
        \textbf{Excellent} \linebreak A+, A, A- &
		\textbf{Très bien}\hfill\hfill \linebreak B+, B, B- & 
		\textbf{Bien} \linebreak C+, C, C- & 
		\textbf{Passable} \linebreak D+, D &
		\textbf{Échec} \linebreak E \\ [0.5ex] 
        \hline\hline
		% Body ------------
		% A+ : 91\% et plus \hfill\hfill \linebreak A : 88 à 90\% \hfill\hfill \linebreak A- : 85 à 87\% \hfill\hfill & 
		% B+ : 82 à 84\% \hfill\hfill \linebreak B : 79 à 81\% \hfill\hfill \linebreak B- : 76 à 78\% \hfill\hfill & 
		% C+ : 73 à 75\% \hfill\hfill \linebreak C : 70 à 72\% \hfill\hfill \linebreak C- : 67 à 69\% \hfill\hfill & 
		% D+ : 64 à 66\% \hfill\hfill \linebreak D : 60 à 63\% \hfill\hfill & 
		% E : 0 à 59\% \hfill \\ 
		% \hline
		% \hline
		% Footer ------------
        \multicolumn{2}{l}{W : échec par abandon\hfill\hfill \linebreak
            AB* : abandon\hfill\hfill \linebreak} &
        \multicolumn{2}{l}{
            IN** : Incomplet\hfill\hfill \linebreak 
            R : réussite\hfill\hfill} \\
		% \hline
		\hline
		\end{tabular}
		\caption*{* La mention AB est consignée seulement si l’étudiante ou l’étudiant abandonne le cours avant la date
		limite d'abandon. Si la date limite d’abandon est dépassée, la mention au relevé de notes de l’étudiante
		ou de l’étudiant sera W.\\
		** La metion IN est utilisée au relevé de notes pour les activités
		pédagogiques lorsque, pour des motifs acceptés par la faculté ou le
		centre universitaire de formation, l’étudiante ou l’étudiant n’a pas
		satisfait à toutes les exigences. Est remplacée par la note W (échec
		par abandon) au relevé de notes du trimestre au cours duquel prend fin
		le délai accordé si l’activité n’a pas été complétée.}
		\end{table}
	\end{center}

    La note finale du cours sera remise au plus tard deux semaines après le
	dépôt de l'évaluation finale.

    \subsection*{Appréciation de la qualité de la langue}

    En conformité avec l'article 17 du règlement facultaire d'évaluation
    des apprentissages des étudiantes et des étudiants, la qualité du français
    écrit dans l’évaluation peut être pis en compte. Tout travail non conforme
    aux exigences quant à la qualité du français écrit et aux normes de
    présentation  peut retourné à l'étudiante ou à l'étudiant et peut aussi
    entrainer la perte de points pour une mauvaise qualité du français écrit.
    La qualité du français peut compter jusqu'à 5\% des points alloués à
    l'évaluation.

	

\end{document}
